\documentclass{RJWThesis}

\usepackage{graphicx}
\usepackage{harvard}

\title{First Year Report: \\ Fixed Lag Particle Filtering for Target Tracking}
\author{Pete Bunch}




\begin{document}

\maketitle
\tableofcontents

\chapter{Introduction}
Everyone loves to track things. Yay! 

\chapter{Literature Review}
\section{Particle Filtering}
\subsection{Bayes of our lives - what is filtering?}
Many tasks in signal processing, science in general, and indeed life, require us to make some estimate of an unknown quantity from indirect, incomplete, or inaccurate observations. By constructing a model to explain how these observations depend on the underlying state, we can infer something about that state. We will express this observation model in terms of a likelihood function:

\begin{equation}
P(Y|X)
\label{eq:LH}
\end{equation}

where $X$ is the state and $Y$ the observations. This is not the whole story - in many cases we are not estimating our unknown state ``from scratch''. Previous experience, prejudice, and prior knowledge can also contribute to our estimates. The likelihood and prior terms can be combined through our friend, Bayes rule \cite{Bayes1763, Laplace1774}, to calculate the posterior probability of the state, i.e. the probability of the state given the observations:

\begin{equation}
P(X|Y) = \frac{P(Y|X)P(X)}{P(Y)}
\label{eq:BayesRule}
\end{equation}

This is the basis of the process of inference. Mathematically, we can assign a probability distribution to the state space of $X$. By applying Bayes rule, we are updating our belief about the values of $X$ using the information in $Y$.

Often the quantity in which we are interested, $X$, is changing over time, and we would like to estimate its value at each point in time given only the observations received so far. In this case we will generally assume that the state is Markovian, i.e. that $X_t|X_{t-1}$ is independent of $X_{1:t-2}$, giving us a hidden Markov state-space model. This is the traditional filtering problem. By constucting a model for the evolution of the unknown state, we can now derive our prior information from our estimate at the previous time step. In discrete time, we now write:

\begin{equation}
P(X_t|Y_{1:t}) = \frac{\int P(Y_t|X_t)P(X_t|X_{t-1})P(X_{t-1}|Y_{1:t-1}) dX_{t-1}}{P(Y_t|Y_{1:t-1})}
\label{eq:SeqBayesRule}
\end{equation}

where the subscript indicates the time and ranges are notated by $:$ in the MATLAB style. $P(X_t|Y_{1:t})$ is called the filtering distribution. Equation~\ref{eq:SeqBayesRule} describes the ideal Bayesian filter.

Instead of marginalising the previous state, we may sometimes want to consider the joint state distribution over all time instances. This may similarly be expanded as:

\begin{equation}
P(X_{1:t}|Y_{1:t}) = \frac{P(Y_t|X_t)P(X_t|X_{t-1})P(X_{1:t-1}|Y_{1:t-1})}{P(Y_t|Y_{1:t-1})}
\label{eq:JointSeqBayesRule}
\end{equation}

The filtering distribution may then be obtained by marginalising out the previous states.

So far, we have expressed the problem entirely in terms of distributions. The same models may be expressed in terms of difference equations of random variables. In the most general form:
\begin{equation}
X_t = f_t(X_{t-1}, V_t)
\label{eq:FilterEq1}
\end{equation}
\begin{equation}
Y_t = g_t(X_t, W_t)
\label{eq:FilterEq2}
\end{equation}

where $f_t$ and $g_t$ are known deterministic functions and $V_t$ and $W_t$ and random variables, known as the process and observation noise respectively.

\subsection{Keep Kalman carry on - the Kalman filter and its extensions}
\subsection{The basic filter}

In a few simple cases the filtering set-up permits the derivation of closed form posterior distributions at each time instant. Most notably, the Kalman filter (KF) \cite{Kalman1960} is an analytic filter for models with continuous state and observation variables, in which both transition and observation models are linear transformations with Gaussian innovations.

\begin{equation}
x_t = A x_t + v_t
\label{eq:LinearFilterEq1}
\end{equation}
\begin{equation}
y_t = C x_t + w_t
\label{eq:LinearFilterEq2}
\end{equation}

where $v_t$ and $w_t$ are now Gaussian random variables with zero mean and covariance matrices $Q$ and $R$. We use lower case variables here to emphasise that these are ``nice'', continuous vectors. (As we shall see, our state variable will later be sets or lists).

Kalman's solution for the linear-Gaussian case is given by:

\begin{equation}
P(x_t|y_{1:t}) = \mathcal{N}(x_t|\mu_t, \Sigma_t )
\label{eq:KF}
\end{equation}
\begin{equation}
P(x_t|y_{1:t-1}) = \mathcal{N}(x_t|\hat{\mu}_t, \hat{\Sigma}_t )
\label{eq:KFp}
\end{equation}
where $\mu_t$, $\Sigma_t$, etc. are given by the following recursions.

Time Updates:
\begin{equation}
\hat{\mu}_t = A \mu_{t-1}
\label{eq:KFTime1}
\end{equation}
\begin{equation}
\hat{\Sigma}_t = A \Sigma_{t-1} A^{T} + Q
\label{eq:KFTime2}
\end{equation}

Measurement Updates:
\begin{equation}
z_t = y_t - C \hat{\mu_t}
\label{eq:KFMeas1}
\end{equation}
\begin{equation}
S_t = C \hat{\Sigma}_t C^{T} + R
\label{eq:KFMeas2}
\end{equation}
\begin{equation}
K_t = \hat{\Sigma}_t C^{T} S_t^{-1}
\label{eq:KFMeas3}
\end{equation}
\begin{equation}
\mu_t = \hat{\mu}_t + K_t z_t
\label{eq:KFMeas4}
\end{equation}
\begin{equation}
\Sigma_t = (I - K_t C) \hat{\Sigma}_t
\label{eq:KFMeas5}
\end{equation}

The KF is delightful because it not only provides us with a closed-form analytic solution, but the complexity of that solution does not increase as we receive additional measurements. This is a consequence of the fact that the Gaussian distribution is its own conjugate prior. Unfortunately, no other such convenient cases have been discovered \cite{Daum2005}. Analytic solutions to non-linear, non-Gaussian filtering problems generally require unacceptable conditions, such as zero process noise $Q=0$ \cite{Daum2005}.

\subsection{The extended filter}

Given the loveliness of the KF, the instinct when faced by an intractable non-linear filtering problem is to linearise it. This produces the Extended Kalman Filter (EKF), and is achieved by replacing the $A$ and $C$ matrices in equations~\ref{eq:KFTime1} through~\ref{eq:KFMeas5} above with Jacobians:

\begin{equation}
A_t = \left . \frac{\partial f}{\partial x_t} \right \vert _{\mu_{t-1}}
\label{eq:EKF1}
\end{equation}
\begin{equation}
C_t = \left . \frac{\partial g}{\partial x_t} \right \vert _{\hat{\mu}_t}
\label{eq:EKF2}
\end{equation}

\subsection{The Kalman smoother}

The KF gives us an optimum estimate of $P(x_t|y_{1:t})$. However, once more data has arrived, we can improve this estimate. For a given set of data, $y_{1:T}$, we can estimate the optimum estimates for all previous state distributions, $P(x_{1:T}| y_{1:T})$ using a Rauch-Tung-Striebel (RTS) smoother, \cite{Rauch1965}. This begins with a normal KF, followed by a backward filtering pass which propagates information to earlier time instances. This backward pass is implement by the following recursions:

\begin{equation}
\tilde{\mu}_t = \mu_{t} + \Sigma_t A^T \hat{\Sigma}_{t+1}^{-1} (\tilde{\mu}_{t+1} - \hat{\mu}_{t+1})
\label{eq:}
\end{equation}
\begin{equation}
\tilde{\Sigma}_t = \Sigma_{t} + [\Sigma_t A^T \hat{\Sigma}_{t+1}^{-1}] (\tilde{\Sigma}_{t+1} - \hat{\Sigma}_{t+1}) [\Sigma_t A^T \hat{\Sigma}_{t+1}^{-1}]^T
\label{eq:}
\end{equation}

giving us

\begin{equation}
P(x_t|Y_{1:T}) = \mathcal{N}(x_t|\tilde{\mu}_t, \tilde{\Sigma}_{t})
\label{eq:}
\end{equation}

For a full derivation, see \cite{Rauch1965}. There exist other ways to implement Kalman smoothing in a fixed-interval sense, such as the forward-backward smoother, and in a fixed-lag sense, but they will not be used in this work.

\subsection{Tough as old bootstraps - the traditional approach}
In general we will not be so lucky as to have a problem with linear-Gaussian dynamics. In this case, a particle filter (PF) may be the best alternative. With a PF, we approximate a probability distribution with a set of (weighted) samples drawn from that distribution.

\begin{equation}
P(X) \approx \frac{1}{N} \sum_m{W^{(m)} \delta_{X} (x^{(m)})}
\label{eq:ParticleApprox}
\end{equation}

where $\delta (x)$ represents a unit probability point mass at a point $x$, and $\sum_m{W^{(m)}}=1$.

Consider the posterior joint state distribution of equation~\ref{eq:JointSeqBayesRule}. 



\subsection{MCMC in da house - series vs. parallel}
S-MCMC (Godsill) approach

\section{Approaches to Tracking}
A brief overview
JPDAF
MHT
Brief mention of PFs

\chapter{Fixed Lag Estimation for Tracking}
\section{Why do now what we can do later?}



\section{A mathematical framework for fixed lag estimation}
Description of the BDS method \cite{Doucet2006}
\section{Fixed lag estimation for tracking}
Application to tracking
\subsection{Proposal Distributions}


\chapter{Fixed Lag Particle Filters}
\section{Sequential Importance Sampling and Resampling}
\subsection{Coping with dimensionality}
\section{Markov Chain Monte Carlo}
\section{Marginalised Particle Filters}

\chapter{Plan Of Future Research}
radar, insects, crowds, cells, jump-diffusions, cursor tracking...

\bibliographystyle{unsrt}
\bibliography{D:/pb404/Bibtex/OTbib}

\end{document}