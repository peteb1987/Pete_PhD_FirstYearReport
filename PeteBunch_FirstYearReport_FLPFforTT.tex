\documentclass{RJWThesis}

\usepackage{amsmath}
\usepackage{IEEEtrantools}

\usepackage{graphicx}
\usepackage{harvard}

\title{Fixed Lag Particle Filtering for Target Tracking}
\subtitle{A report on the first year's progress}
\author{Pete Bunch}
\date{August 2011}




\begin{document}

\maketitle
\tableofcontents


\chapter{Introduction}
Everyone loves to track things. Yay! 




\chapter{Particle Methods} \label{chap:basics}
\section{Bayes of our lives - what is filtering?}
Many tasks in signal processing, science in general, and indeed life, require us to make some estimate of an unknown quantity from indirect, incomplete, or inaccurate observations. By constructing a model to explain how these observations depend on the underlying state, we can infer something about that state. We will express this observation model in terms of a likelihood function:

\begin{equation}
P(Y|X)
\label{eq:LH}
\end{equation}

where $X$ is the state and $Y$ the observations. This is not the whole story - in many cases we are not estimating our unknown state ``from scratch''. Previous experience, prejudice, and prior knowledge can also contribute to our estimates. The likelihood and prior terms can be combined through our friend, Bayes rule \cite{Bayes1763, Laplace1774}, to calculate the posterior probability of the state, i.e. the probability of the state given the observations:

\begin{equation}
P(X|Y) = \frac{P(Y|X)P(X)}{P(Y)}
\label{eq:BayesRule}
\end{equation}

This is the basis of the process of inference. Mathematically, we can assign a probability distribution to the state space of $X$. By applying Bayes rule, we are updating our belief about the values of $X$ using the information in $Y$.

Often the quantity in which we are interested, $X$, is changing over time, and we would like to estimate its value at each point in time given only the observations received so far. In this case we will generally assume that the state is Markovian, i.e. that $X_t|X_{t-1}$ is independent of $X_{1:t-2}$, giving us a hidden Markov state-space model. This is the traditional filtering problem. By constucting a model for the evolution of the unknown state, we can now derive our prior information from our estimate at the previous time step. In discrete time, we now write:

\begin{equation}
P(X_t|Y_{1:t}) = \frac{\int P(Y_t|X_t)P(X_t|X_{t-1})P(X_{t-1}|Y_{1:t-1}) dX_{t-1}}{P(Y_t|Y_{1:t-1})}
\label{eq:SeqBayesRule}
\end{equation}

where the subscript indicates the time and ranges are notated by $:$ in the MATLAB style. $P(X_t|Y_{1:t})$ is called the filtering distribution. Equation~\ref{eq:SeqBayesRule} describes the ideal Bayesian filter.

Instead of marginalising the previous state, we may sometimes want to consider the joint state distribution over all time instances. This may similarly be expanded as:

\begin{equation}
P(X_{1:t}|Y_{1:t}) = \frac{P(Y_t|X_t)P(X_t|X_{t-1})P(X_{1:t-1}|Y_{1:t-1})}{P(Y_t|Y_{1:t-1})}
\label{eq:JointSeqBayesRule}
\end{equation}

The filtering distribution may then be obtained by marginalising out the previous states.

So far, we have expressed the problem entirely in terms of distributions. The same models may be expressed in terms of difference equations of random variables. In the most general form:
\begin{equation}
X_t = f_t(X_{t-1}, V_t)
\label{eq:FilterEq1}
\end{equation}
\begin{equation}
Y_t = g_t(X_t, W_t)
\label{eq:FilterEq2}
\end{equation}

where $f_t$ and $g_t$ are known deterministic functions and $V_t$ and $W_t$ and random variables, known as the process and observation noise respectively.

\section{Keep Kalman carry on - the Kalman filter and its extensions}
\subsection{The basic filter}

In a few simple cases the filtering set-up permits the derivation of closed form posterior distributions at each time instant. Most notably, the Kalman filter (KF) \cite{Kalman1960} is an analytic filter for models with continuous state and observation variables, in which both transition and observation models are linear transformations with Gaussian innovations.

\begin{equation}
x_t = A x_t + v_t
\label{eq:LinearFilterEq1}
\end{equation}
\begin{equation}
y_t = C x_t + w_t
\label{eq:LinearFilterEq2}
\end{equation}

where $v_t$ and $w_t$ are now Gaussian random variables with zero mean and covariance matrices $Q$ and $R$. We use lower case variables here to emphasise that these are ``nice'', continuous vectors. (As we shall see, our state variable will later be sets or lists).

Kalman's solution for the linear-Gaussian case is given by:

\begin{equation}
P(x_t|y_{1:t}) = \mathcal{N}(x_t|\mu_t, \Sigma_t )
\label{eq:KF}
\end{equation}
\begin{equation}
P(x_t|y_{1:t-1}) = \mathcal{N}(x_t|\hat{\mu}_t, \hat{\Sigma}_t )
\label{eq:KFp}
\end{equation}
where $\mu_t$, $\Sigma_t$, etc. are given by the following recursions.

Time Updates:
\begin{equation}
\hat{\mu}_t = A \mu_{t-1}
\label{eq:KFTime1}
\end{equation}
\begin{equation}
\hat{\Sigma}_t = A \Sigma_{t-1} A^{T} + Q
\label{eq:KFTime2}
\end{equation}

Measurement Updates:
\begin{equation}
z_t = y_t - C \hat{\mu_t}
\label{eq:KFMeas1}
\end{equation}
\begin{equation}
S_t = C \hat{\Sigma}_t C^{T} + R
\label{eq:KFMeas2}
\end{equation}
\begin{equation}
K_t = \hat{\Sigma}_t C^{T} S_t^{-1}
\label{eq:KFMeas3}
\end{equation}
\begin{equation}
\mu_t = \hat{\mu}_t + K_t z_t
\label{eq:KFMeas4}
\end{equation}
\begin{equation}
\Sigma_t = (I - K_t C) \hat{\Sigma}_t
\label{eq:KFMeas5}
\end{equation}

The KF is delightful because it not only provides us with a closed-form analytic solution, but the complexity of that solution does not increase as we receive additional measurements. This is a consequence of the fact that the Gaussian distribution is its own conjugate prior. Unfortunately, no other such convenient cases have been discovered \cite{Daum2005}. Analytic solutions to non-linear, non-Gaussian filtering problems generally require unacceptable conditions, such as zero process noise $Q=0$ \cite{Daum2005}.

\subsection{The extended filter}

Given the loveliness of the KF, the instinct when faced by an intractable non-linear filtering problem is to linearise it. This produces the Extended Kalman Filter (EKF), and is achieved by replacing the $A$ and $C$ matrices in equations~\ref{eq:KFTime1} through~\ref{eq:KFMeas5} above with Jacobians:

\begin{equation}
A_t = \left . \frac{\partial f}{\partial x_t} \right \vert _{\mu_{t-1}}
\label{eq:EKF1}
\end{equation}
\begin{equation}
C_t = \left . \frac{\partial g}{\partial x_t} \right \vert _{\hat{\mu}_t}
\label{eq:EKF2}
\end{equation}

\subsection{The Kalman smoother}

The KF gives us an optimum estimate of $P(x_t|y_{1:t})$. However, once more data has arrived, we can improve this estimate. For a given set of data, $y_{1:T}$, we can estimate the optimum estimates for all previous state distributions, $P(x_{1:T}| y_{1:T})$ using a Rauch-Tung-Striebel (RTS) smoother, \cite{Rauch1965}. This begins with a normal KF, followed by a backward filtering pass which propagates information to earlier time instances. This backward pass is implement by the following recursions:

\begin{equation}
\tilde{\mu}_t = \mu_{t} + \Sigma_t A^T \hat{\Sigma}_{t+1}^{-1} (\tilde{\mu}_{t+1} - \hat{\mu}_{t+1})
\label{eq:}
\end{equation}
\begin{equation}
\tilde{\Sigma}_t = \Sigma_{t} + [\Sigma_t A^T \hat{\Sigma}_{t+1}^{-1}] (\tilde{\Sigma}_{t+1} - \hat{\Sigma}_{t+1}) [\Sigma_t A^T \hat{\Sigma}_{t+1}^{-1}]^T
\label{eq:}
\end{equation}

giving us

\begin{equation}
P(x_t|Y_{1:T}) = \mathcal{N}(x_t|\tilde{\mu}_t, \tilde{\Sigma}_{t})
\label{eq:}
\end{equation}

For a full derivation, see \cite{Rauch1965}. There exist other ways to implement Kalman smoothing in a fixed-interval sense, such as the forward-backward smoother, and in a fixed-lag sense, but they will not be used in this work.

\section{Tough as old bootstraps - a review of particle filter}
In general we will not be so lucky as to have a problem with linear-Gaussian dynamics. In this case, a particle filter (PF) may be the best alternative. The PF is based on the idea that if we cannot represent a probability distribution analytically then we may approximate it as a collection of discrete samples or ``particles'' drawn from the distribution. Often, it will not even be possible to sample from the desired distribution. We therefore use importance sampling (IS) or Markov chain Monte Carlo methods to generate these samples. The principle of a particle filter is that we recursively generate such a set of particles to approximate $P(X_{1:t}|Y_{1:t})$ given a previous set representing $P(X_{1:t-1}|Y_{1:t-1})$.

The particle filter 

\section{Particular details - mathematical basics of the particle filter}
In general we will not be so lucky as to have a problem with linear-Gaussian dynamics. In this case, a particle filter (PF) may be the best alternative. With a PF, we approximate a probability distribution with a set of (weighted) samples drawn from that distribution.

\begin{equation}
P(X) \approx \frac{1}{N} \sum_m{W^{(m)} \delta_{X} (x^{(m)})}
\label{eq:ParticleApprox}
\end{equation}

where $\delta (x)$ represents a unit probability point mass at a point $x$, and $\sum_m{W^{(m)}}=1$.

Consider the posterior joint state distribution of equation~\ref{eq:JointSeqBayesRule}. 



\section{MCMC in da house - series vs. parallel}
S-MCMC (Godsill) approach




\chapter{Approaches to Tracking}

\section{Why do we track?}
In a target tracking scenario, the aim is to trace the trajectory of an object over time from a set of discrete observations. In the previous chapter we saw how we can carry out inference tasks using the recursions of the Kalman filter or the particle filter. Surely now we are done!? What else remains? The additional difficulty in target tracking is the nature of the observation process. Our imperfect sensors may not detect every target present in the scene in every scan. Furthermore, there may be some number of false alarms arising from sensor errors or clutter. Our task thus becomes three-fold: firstly to detect what targets are present in the scene, secondly to work out which observations were generated by each target, and finally to estimate the states of the targets. The basic Kalman and particle filters address only the third of these tasks.

Research in target tracking emerged from military applications such as radar and sonar. However, similar problems emerge in many diverse branches of science: the tracking of people, vehicles or animals in video sequences; of molecules or cells in microscopy data; or of notes in a piece of music. Each can be reduced to a similar underlying model. In this chapter we will formulate such a basic model, keeping it as general as possible rather than focusing on any specific application.


\section{Models for tracking}
In this section we introduce the basic models and notation for the object tracking problem which will be used throughout this report.

For our first assumptions, we will consider our targets to exist at singular points in space, and that targets move independently of each other. Each target will have be characterised by a state vector $x_t$. This state is composed of continuous-valued coordinates and evolves according to a hidden Markov model (HMM).

\begin{equation}
x_t = f(x_{t-1}, v_t)
\end{equation}

where $v_t$ is a random vector.

The targets are observed by a sensor which detects each with a probability $P_D$. If detected, the sensor returns a point observation at a location $y_t$ given by:

\begin{equation}
y_t = g(x_t, w_t)
\end{equation}

where $w_t$ is another random vector, independent of $v_t$.

In addition to the target-originating observations, the sensor also detects a number of false alarms. Throughout this report we assume that these are generated by a Poisson process, with uniform intensity over the observation area.

We denote the set of targets present as $X_t = \{x_{1,t}, x_{2,t}, ... , x_{K_t, t} \}$, and the set of observations as $Y_t = \{y_t^{(1)}, y_t^{(2)}, ... , y_t^{(M_t)} \}$.



\subsection{Data association}

In order to conduct inference on our multi-target system, we would like to calculate the likelihood, $P(Y_t|X_t)$. In order to evaluate this term, we need to hypothesise some assignment between the observations and the process which generated them, whether a particular target or clutter. We introduce an association variable for each target, $\lambda_{j,t}$, which indicates which of the observations in that frame was generated by this target. If the target is not detected, then $\lambda_t$ is set to 0. We denote the set $\Lambda_t = \{\lambda_{1,t}, \lambda_{2,t}, ... , \lambda_{K_t, t} \}$. As each observations is generated by one target or clutter, no two elements of $\Lambda_t$ may take the same value, unless 0.

We can now expand the likelihood as:

\begin{equation}
P(Y_t|X_t) = \sum_{\Lambda_t} \prod_{j=1}^{K_t} P(y_t^{(\lambda_{j,t})}|x_{j,t})
\label{eq:}
\end{equation}

where the summation is over all feasible values of $\Lambda_t$. The number of terms in this summation is $M_t (M_t-1) ... (M_t-K_t)$, which may be prohibitively large if there are many targets or observations in the scene.

If it is necessary to calculate the likelihood very often, as in a particle filter, it may be preferable to estimate $\Lambda_t$ instead of marginalising it.

\section{Algorithms for tracking}
JPDAF
MHT
MCMCDA
Particle filters. Especially Vermaak
RBPF




\chapter{Fixed Lag Estimation for Tracking}
\section{Why do now what we can do later?}
In chapter~\ref{chap:basics}, we looked at filtering methods, that is, estimation of a hidden state $X_t$ given observations up until this time $Y_{1:t}$. For such a task, Kalman filters and particle filters are our tools of choice. However, for difficult target tracking problems, there may simply not be enough information available to make a good estimate. Consider a target with a low detection probability and high clutter. There is likely to be multiple observations with which the target could be associated in each frame, or the target could disappear entirely for a period. The result is that there is a very large number of possible routes through the observation space. A conventional particle filter will have to maintain particles on each of the possible routes until such time as they become negligibly unlikely. The correct route may at first only have a few particles on it, and when it is identified as the correct route in later frames the particle diversity may be poor. In the worst case, the correct route may at first look so unlikely (e.g. consecutive missed detections), that no particles follow it, and the track is lost.

This problem can be addressed by allowing not only a proposal of a new state $X_t$ on the end of each path, but also the re-proposal of the preceeding states as well. Thus, particles which have followed an incorret route may be ``redirected'', to follow a more probable course. This helps maintain particle diversity. If few or no particles follow the correct route at first, particles may be diverted later once the course becomes more apparent.

In an ordinary particle filter, we estimate the joint posterior, $P(X_{1:t}|Y_{1:t})$ at each time instant. Thus, our estimates of the state distribution at previous times are updated at each step, but only by resampling. Diversity in the marginal distributions for previous states only decreases as time progresses. Conversely, in using such a system of re-proposals we allow diversity to be maintained in previous states. To limit the computational complexity, we constrain the region in which re-proposals may occur to a lagging window of $L$ time steps.



\section{A mathematical framework for fixed lag estimation}
We now consider a mathematical framework for fixed lag estimation. This method was devised in \cite{Doucet2006} and \cite{Briers2006}.

As before, the target posterior distribution in which we are interested is the familiar $P(X_{1:t}|Y_{1:t})$. However, the proposal mechanism now becomes more complex, because we will be replacing states in an existing particle. We first propose a particle from which to take the state ``history'', that is $X_{1:t-L}$. This may be chosen from the particle approximation from any of the previous $L$ processing steps. However, we get more of the path than we need, because each particle at lag $s$ is a set of states $X_{1:t-s}$. The final $L-s$ states will be replaced when by a new ``tip'', $X'_{t-L+1:t}$, drawn from an importance distribution. The complete proposal is thus

\begin{equation}
\{X_{1:t-L}, X'_{t-L+1:t}\} \sim q(s) \int q(X_{1:t-s}|Y_{1:t-s}) q(X'_{t-L+1}|X_{1:t-s}, Y_{t-L+1:t}) dX_{t-L+1:t-s}
\label{eq:DumbFLProposal1}
\end{equation}

where $q(X_{1:t-s}|Y_{1:t-s})$ is a proposal distribution using the arbitrarily-weighted particles from $\hat{P}(X_{1:t-s}|Y_{1:t-s})$. We cannot evaluate this. In general, the integral will be intractable. If we restrict our proposals to depend only on the history, i.e. use $q(X_{t-L+1}|X_{1:t-L}, Y_{t-L+1:t})$, then the proposal becomes

\begin{equation}
\{X_{1:t-L}, X'_{t-L+1:t}\} \sim q(s) \hat{P}(X_{1:t-L}|Y_{1:t-s}) q(X'_{t-L+1}|X_{1:t-s}, Y_{t-L+1:t})
\label{eq:DumbFLProposal2}
\end{equation}

% equation~\ref{eq:DumbFLProposal} simplifies to $q(s) \hat{P}(X_{1:t-L}|Y_{1:t-s}) q(X_{t-L+1}|X_{1:t-L}, Y_{t-L+1:t})$. The importance weight is then given by (corresponding MCMC acceptance probabilities will be given by a ratio of two such terms):

%\begin{equation}
%W_t^{(m)} = \frac{P(Y_{t-L+1:t}|X'^{(m)}_{t-L+1:t}) P(X'^{(m)}_{t-L+1:t}|X^{(m)}_{t-L})}{P(Y_{t-s+1:t}|Y_{1:t-s}) P(Y_{t-L+1:t}|X^{(m)}_{t-L}) q(s) q(X'^{(m)}_{t-L+1:t}|X^{(m)}_{1:t-L}, Y_{t-L+1:t})}
%\label{eq:}
%\end{equation}

However, with this proposal, in order to evaluate importance weights or acceptance probabilities we still need to calculate

\begin{equation}
P(X_{1:t-L}|Y_{1:t-s}) = \frac{P(Y_{t-L+1:t-s}|X_{t-L}^{(m)}) P(X_{1:t-L}|Y_{1:t-L} }{P(Y_{t-L+1:t-s}|Y_{1:t-L})}
\label{eq:}
\end{equation}

The problematic term is:

\begin{equation}
P(Y_{t-L+1:t}|X^{(m)}_{t-L}) = \int P(Y_{t-L+1:t}|X^{(m)}_{t-L}, X_{t-L+1:t}) P(X_{t-L+1:t}|X^{(m)}_{t-L}) dX_{t-L+1:t-s}
\label{eq:}
\end{equation}

In general, this will be intractable. The exception is the case where $s=L$, but this will in general not work well - such a requirement implies ignoring all the estimation we have done since the start of the window!

The solution to this tractability problem proposed in \cite{Doucet2006} is to augment the dimension of the target distribution to include the discarded tracks. The new target is:

\begin{equation}
P(X_{1:t-L}, X'_{t-L+1:t}|Y_{1:t}) \rho(X_{t-L+1:t-s}|X_{1:t-L}, X'_{t-L+1:t})
\label{eq:}
\end{equation}

Once a particle approximation has been generated for this target the required posterior distribution may be obtained by marginalisation. As we are simply going to discard the old track sections $X_{t-L+1:t-s}$, the choice of $\rho(.)$ does not alter the distribution of the posterior. However, the variance of weights/acceptance probabilities may be affected.

With an expanded space for the target distribution there is no need to marginalise the previous states. The proposal distribution now becomes:

\begin{equation}
\{X_{1:t-s}, X'_{t-L+1:t}\} \sim q(s) q(X_{1:t-s}|Y_{1:t-s}) q(X'_{t-L+1}|X_{1:t-s}, Y_{t-L+1:t})
\label{eq:ExtendedFLProposal}
\end{equation}

\subsection{Artificial conditionals}

We use $\rho(.)=q(.)$. Why?

\subsection{History Proposals}

``Conservative'' resampling as a history proposal.

\section{Applying the fixed lag method to the tracking model}
Erm... i've kinda said what I was going to say here in the previous section

\subsection{Importance distributions}

The dimensionality of a fixed lag particle filter will generally be very large. Consider a problem where each target has a 4-dimensional kinematic state (postion and velocity in $x$ and $y$) and an observation-association index. This gives us five dimensions per target per time step. If we use a lag window with length $L=5$ and 5 targets we will have a 125 dimensional state. If we used a basic, bootstrap approach to such a problem, the chances of even a single particle following the correct path are negligible. Instead we must exploit the strong correlations between states arising from the structure of the problem. We first factorise the proposal thus:

%First we consider the importance distribution for a single target and a lag of $L$. The ``optimal'' distribution for this is

%\begin{multline}
%q(x_{t-L+1:t}, \lambda_{t-L+1:t}|x_{t-L}, Y_{t-L+1:t}) = P(x_{t-L+1:t}, \lambda_{t-L+1:t}|x_{t-L}, Y_{t-L+1:t}) = \\
%\frac{P(Y_{t-L+1:t}|x_{t-L+1:t}, \lambda_{t-L+1:t}) P(X_{t-L+1:t}, \lambda_{t-L+1:t}|x_{t-L})}{P(Y_{t-L+1:t}|x_{t-L})}
%\label{eq:}
%\end{multline}

%\begin{multline}
%q(x_{t-L+1:t}, \lambda_{t-L+1:t}|x_{t-L}, Y_{t-L+1:t}) = \\
%q(\lambda_t|x_{t-L}, Y_t) \prod_{k=t-L+1}^{t-1} {q(\lambda_k| \lambda_{k+1:t}, Y_{k:t}, x_{t-L})} \nonumber \\
%q(x_t|x_{t-L}, \lambda_{t-L+1:t}, Y_{t-L+1:t}) \prod_{k=t-L+1}^{t-1} {q(x_k|x_{t-L}, x_{k+1}, \lambda_{t-L+1:k}, Y_{t-L+1:k})} \nonumber
%\label{eq:}
%\end{multline}

\begin{IEEEeqnarray}{rCl}
q(x_{t-L+1:t}, \lambda_{t-L+1:t}|x_{t-L}, Y_{t-L+1:t}) & & \nonumber \\
 & = & q(\lambda_{t-L+1:t}|x_{t-L}, Y_{t-L+1:t}) \nonumber \\
 & = & q(x_{t-L+1:t}|x_{t-L}, \lambda_{t-L+1:t}, Y_{t-L+1:t})
\label{eq:}
\end{IEEEeqnarray}

%This can be sampled sequentially. We start with the last association variable, $\lambda_t$, then proceed backwards in time through $\lambda_{t-L+1:t-1}$. Once the association variables are set, the kinematic states may be sampled similarly using the method suggested in \cite{Doucet2006} and originally in \cite{Chib1996}, in which we first propose $x_t$ then sequentially backwards through $x_{t-L+1:t-1}$.

Thus we can sample first the association variables then the state variables. Each of these terms can be further factorised over time, as we shall see. If we were to replace each of the factors with its corresponding posterior distribution then this factorsiation would recreate the ``optimal'' importance distribution.% In general we will not be able to either sample from or calculate such a posterior. The exception is the case where we have linear-Gaussian dynamics.

\subsubsection{Association proposals}

The proposal for the association variables is a discrete distribution over the possible observations with which the target could be associated. Within the factorisation above, the ``optimal'' form of the proposal is:

\begin{IEEEeqnarray}{rCl}
q(\lambda_{t-L+1:t}|x_{t-L}, Y_{t-L+1:t}) & & \nonumber \\
 & \propto & P(Y_{t-L+1:t}|\lambda_{t-L+1:t},x_{t-L}) P(\lambda_{t-L+1:t}|x_{t-L}) \nonumber \\
 & = & \prod_{\substack{k=1:L\\tt=t-L+k}} P(Y_{tt}|Y_{tt+1:t} \lambda_{tt:t}, x_{t-L}) P(\lambda_{tt}) \nonumber \\
 & = & \prod_{\substack{k=1:L\\tt=t-L+k}} \int P(Y_{tt}|x_{tt}, \lambda_k) P(x_{tt}|Y_{tt+1:t}, \lambda_{tt+1:t}, x_{t-L}) dx_{tt} P(\lambda_{tt})
\label{eq:GeneralAssocProp}
\end{IEEEeqnarray}

This suggests a convenient sequential sampling procedure, starting with $\lambda_t$ and working backwards in time. The nomalisation constant is not known, but as this is discrete distribution we can enforce normalisation by dividing by the sum.

First we consider a factor from this expression with $\lambda_tt=1$, i.e. a proposal that in a particular frame the target is not detected. In this case, the observation density is independent of the state of the target, and we have:

\begin{equation}
P(Y_{tt}|Y_{tt+1:t} \lambda_{tt:t}, x_{t-L}) P(\lambda_{tt}) = V^{-1} P(\lambda_t=0)
\label{eq:}
\end{equation}

When the target is detected, the factors of the proposal ditribution of equation~\ref{eq:GeneralAssocProp} may calculated analytically for the linear-Gaussian case. For other cases, the EKF approximations may be used. As this is only a proposal distribution, such an approximation will not affect the distribution of the particle distribution generated. The state distribution term over $x_{tt}$ is problematic, requiring $k$ integrals to calculate. Although this is will be analytic with Gaussian dynamics, its complexity may become unmanageable. This term represents the probability of the state given the set of observations associated with this target at later times. We can render this calculation more manageable by replacing the whole set of future observations with just one.

\begin{equation}
P(x_k|Y_{k+1:t}, \lambda_{k+1:t}, x_{t-L}) \approx P(x_k|Y_{k+d}, \lambda_{k+d}, x_{t-L})
\label{eq:}
\end{equation}

For cases with low observation noise, where an observation gives us significant information about the state of the target, this substitution will have little effect. Later observations cannot add much additional information. Again, as this is a proposal distribution, such a substitution will not affect the validity of the resulting particle distribution.

In general we will use $d=1$, as the closest observation in time will give us the most information about $x_{tt}$. However, if the target is not detected at time $tt+1$, then we can increase $d$ to find the next detection of the target.

Using this approximation, for $\lambda_{t-L+k} \ne 0$, we have

\begin{equation}
P(Y_{tt}|Y_{tt+1:t} \lambda_{tt:t}, x_{t-L}) P(\lambda_{tt}) \propto \mathcal{N}(y_{tt}^{(\lambda_{tt})}|m_{tt}, S_{tt})
\end{equation}

where usually

\begin{equation} S_{tt} = [ I - R^{-1} C_{tt} \Sigma_{tt} C_{tt} R^{-1} ]^{-1} \label{eq:} \end{equation}
\begin{equation} m_{tt} = S R^{-1} C_{tt} \Sigma_{tt} [ (A^d)^T C_{tt+d}^T R_d^{-1} y_{tt+d}^{\lambda_{tt+d}} + Q_k^{-1} A^k x_{t-L} ] \label{eq:} \end{equation}
\begin{equation} \Sigma_{tt} = [ C_{tt}^T R^{-1} C_{tt} + (A^d)^T C_{tt+d}^T R_d^{-1} C_{tt+d} A^d + Q_k^{-1}]^{-1} \label{eq:} \end{equation}
\begin{equation} Q_d = \sum_{l=0}^{d-1} {A^l Q (A^l)^T} \label{eq:} \end{equation}
\begin{equation} R_d = R + C_{tt+d} Q_d (C_{tt+d})^T \label{eq:} \end{equation}

We will need a different expression for the case when $tt=t$, because no future associations have yet been proposed. Similarly, if $tt<t$ but the future associations have all been proposed as missed detections, then there are no future observations to guide us, whatever choice of $d$ we use. In these cases we have:

\begin{equation} S_{tt} = R_d \end{equation}
\begin{equation} m_{tt} = C_{tt} A^k x_{t-L} \label{eq:} \end{equation}

For full derivations, see Appendix.

Finally, substituting for the association prior terms, we have:

\begin{equation}
q(\lambda_{t-L+1:t}|x_{t-L}, Y_{t-L+1:t}) \propto \prod_{\substack{k=1:L\\tt=t-L+k}} \begin{cases}
P_D \mathcal{N}(y_{tt}^{(\lambda_{tt})}|m_{tt}, S_{tt}) & \lambda_{tt}=0 \\
(1-P_D) \mu_C V^{-1} & \lambda_{tt} \ne 0 \end{cases}
\label{eq:}
\end{equation}

This gives us a complete sequential mechanism for proposing the asociations.



\subsubsection{State proposals}

Once the associations are fixed, the states can be proposed. When the state space model is linear-Gaussian, we can propose directly from the ``optimal'' importance distribution for the states using the forward-filtering-backward-sampling algorithm of \cite{Chib1996}, as suggested in \cite{Doucet2006}. For nonlinear models, we can use EKF approximations, as for the associations. Once again, we factorise the proposal:

\begin{multline}
q(x_{t-L+1:t}|x_{t-L}, \lambda_{t-L+1:t}, Y_{t-L+1:t}) \\
= P(x_{t-L+1:t}|x_{t-L}, \lambda_{t-L+1:t}, Y_{t-L+1:t}) \\
= P(x_t|\lambda_{t-L+1:t}, Y_{t-L+1:t}, x_{t-L}) \prod_{k=t-L+1}^{t-1} P(x_k|\lambda_{t-L+1:k}, Y_{t-L+1:k}, x_{t-L}, x_{k+1})
\label{eq:}
\end{multline}

where

\begin{equation}
P(x_k|\lambda_{t-L+1:k}, Y_{t-L+1:k}, x_{t-L}, x_{k+1}) \propto P(x_{k+1}|x_k) P(x_k|\lambda_{t-L+1:k}, Y_{t-L+1:k}, x_{t-L})
\label{eq:}
\end{equation}

The distributions $P(x_k|\lambda_{t-L+1:k}, Y_{t-L+1:k}, x_{t-L})$ are given by a Kalman filter, and are Gaussian with mean $\mu_k$ and covariance $\Sigma_k$. Thus the complete state proposal is given by:

\begin{equation}
q(x_{t-L+1:t}|x_{t-L}, \lambda_{t-L+1:t}, Y_{t-L+1:t}) = \mathcal{N}(x_t|\mu_t, \Sigma_t) \prod_{k=t-L+1}^{t-1} \mathcal{N}(x_k|m_k, S_k)
\label{eq:}
\end{equation}

where
\begin{equation}S_k = [ A^T Q^{-1} A + \Sigma^{-1} ]^{-1}\label{eq:}\end{equation}
\begin{equation}m_k = S_k [ A^T Q^{-1} x_{k+1} + \Sigma^{-1} \mu_k ]\label{eq:}\end{equation}




\chapter{Fixed Lag Particle Filters}
\section{Sequential Importance Sampling and Resampling}
\subsection{Coping with dimensionality}
\section{Markov Chain Monte Carlo}
\section{Marginalised Particle Filters}




\chapter{Plan Of Future Research}
radar, insects, crowds, cells, jump-diffusions, cursor tracking...




\bibliographystyle{dcu_noURLs}
\bibliography{D:/pb404/Bibtex/OTbib}

\end{document}