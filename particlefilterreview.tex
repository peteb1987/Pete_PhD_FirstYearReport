In general we will not be so lucky as to have a problem with linear-Gaussian dynamics. In this case, a particle filter (PF) may be the best alternative. The PF is based on the idea that if we cannot represent a probability distribution analytically then we may approximate it as a collection of discrete samples or ``particles'' drawn from the distribution. Often, it will not even be possible to sample from the desired distribution. We therefore use importance sampling (IS) or Markov chain Monte Carlo methods to generate these samples. The principle of a particle filter is that we recursively generate such a set of particles to approximate $P(X_{1:t}|Y_{1:t})$ given a previous set representing $P(X_{1:t-1}|Y_{1:t-1})$.

The particle filter 