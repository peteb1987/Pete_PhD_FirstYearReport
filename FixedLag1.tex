A basic filter, the Kalman filter included, allows us to estimate the latest value of a hidden state given all previous observations. However, if we wait until the next observation arrives, we can often make a much better estimate - we now know where the state is going as well as where it came from. This is the idea exploited by the Kalman smoother. In fact, as we have seen, the regular particle filter gives us an estimate of the entire state history, $X_{1:t}$. However, due to the resampling process, the particle diversity of the state distribution for previous frames falls as we move further back in time.

PICTURE

As well as giving poor estimates for state distributions, a further disadvantage of this falling diversity in a tracking context is a tendency to lose tracks. Suppose that in some frame the majority of particles associate with a particular likely-looking observation and head off in one direction, but that a few frames later it becomes apparent that this is an error, and that the correct route goes another way. With an ordinary SISR particle filter the majority of the particles are now eliminated by resampling. The particle diversity of the state estimate will be very poor, and in a challenging problem there is a good chance that the track will be lost.

PICTURE

Improved estimates for previous state distributions can be made using the resample-move method of \cite{Gilks2001} to rejeuvenate the particle diversity. However, this requires running a Markov chain for each particle on top of the usual importance sampling step, which can be computationally expensive.

ADD MORE COMPARISON WITH RESAMPLE-MOVE. IN SOME SENSE, THEY MUST BE EQUIVALENT - WORK OUT WHAT IT IS.

An alternative framework for fixed lag estimation is presented by \cite{Doucet2006}. Now we propose new values not only for the current state but also the previous states within some time window. The effect of this in our tracking example is that we can ``redirect'' particles which have gone astray along the correct path, maintaining a better particle diversity and reducing the probability of track loss.

