In a target tracking scenario, the aim is to trace the trajectory of an object over time from a set of discrete observations.

filler here

The additional difficulty in target tracking is the nature of the observation process. Our imperfect sensors may not detect every target present in the scene in every scan. Furthermore, there may be some number of false alarms arising from sensor errors or clutter. Our task thus becomes three-fold: firstly to detect what targets are present in the scene, secondly to work out which observations were generated by each target, and finally to estimate the states of the targets. The basic Kalman and particle filters address only the third of these tasks.

Research in target tracking emerged from military applications such as radar and sonar. However, similar problems emerge in many diverse branches of science: the tracking of people, vehicles or animals in video sequences; of molecules or cells in microscopy data; or of notes in a piece of music. Each can be reduced to a similar underlying model. In this chapter we will formulate such a basic model, keeping it as general as possible rather than focusing on any specific application.

Report Structure