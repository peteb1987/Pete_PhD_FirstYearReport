In a target tracking problem, the aim is to trace the trajectory of an object over time from a set of discrete observations. At first glance such a problem may appear no different from a standard case of state estimation to be solved with a Kalman filter (if linear) or particle filter (if not). The additional difficulty in target tracking is the nature of the observation process. Our imperfect sensors may not detect every target present in the scene in every scan. Furthermore, there may be some number of false alarms arising from sensor errors or clutter. Our task thus becomes three-fold: firstly to detect what targets are present in the scene, secondly to work out which observations were generated by each target, and finally to estimate the states of the targets.

Research in target tracking emerged from military applications such as radar and sonar. However, similar problems emerge in many diverse branches of science: the tracking of people, vehicles or animals in video sequences; of molecules or cells in microscopy data; or of notes in a piece of music. Each can be reduced to a similar underlying model.

In this work, a number of new algorithms have been developed for target tracking. These algorithms employ fixed-lag particle filters, which allow previous states to be re-proposed when later observations have been made. Examples have been implemented using both sequential importance sampling with resampling and Markov chain Monte Carlo.

This report is structured as follows:

In chapter 2 we review the relevant literature on tracking and particle filtering for nonlinear estimation.

In chapter 3 we review some mathematical basics which underlie the inference methods used later and set out basic notation.

In chapter 4 we set out the tracking models used and investigate factorisation methods for efficient particle filter implementations.

In chapter 5 we introduce a framework for fixed-lag particle filtering, including the required proposal distributions.

In chapter 6 we present two implementations of fixed-lag particle filters for tracking, one using sequential importance sampling with resampling, the other using Markov chain Monte Carlo.

In chapter 7 we outline additions to the models and algorithms to handle joint detection and tracking of multiple targets.

In chapter 8 we present results and performance evaluations of the trackers.

In chapter 9 we examine a separate project on inference of intended destinations from mouse cursor data.

In chapter 10 a number of ideas for potential future research are set out.