In this section we introduce the basic models and notation for the object tracking problem which will be used throughout this report.

We first assume that our targets exist at singular points in space, and that targets move independently of each other. Each target will have be characterised by a state vector $x_t$. This state is composed of continuous-valued coordinates and evolves according to a hidden Markov model (HMM).

\begin{equation}
x_t = f(x_{t-1}, v_t)
\end{equation}

where $v_t$ is a random vector.

The targets are observed by a sensor which detects each with a probability $P_D$. If detected, the sensor returns a point observation at a location $y_t$ given by:

\begin{equation}
y_t = g(x_t, w_t)
\end{equation}

where $w_t$ is another random vector, independent of $v_t$.

In addition to the target-originating observations, the sensor also detects a number of false alarms. Throughout this report we assume that these are generated by a Poisson process, with uniform intensity over the observation area.

We denote the set of targets present as $X_t = \{x_{1,t}, x_{2,t}, ... , x_{K_t, t} \}$, and the set of observations as $Y_t = \{y_t^{(1)}, y_t^{(2)}, ... , y_t^{(M_t)} \}$.



\subsection{Data association}

In order to conduct inference on our multi-target system, we would like to calculate the likelihood, $P(Y_t|X_t)$. In order to evaluate this term, we need to hypothesise some assignment between the observations and the process which generated them, whether a particular target or clutter. We introduce an association variable for each target, $\lambda_{j,t}$, which indicates which of the observations in that frame was generated by this target. If the target is not detected, then $\lambda_t$ is set to 0. We denote the set $\Lambda_t = \{\lambda_{1,t}, \lambda_{2,t}, ... , \lambda_{K_t, t} \}$. As each observations is generated by one target or clutter, no two elements of $\Lambda_t$ may take the same value, unless 0.

We can now expand the likelihood as:

\begin{equation}
P(Y_t|X_t) = \sum_{\Lambda_t} \prod_{j=1}^{K_t} P(y_t^{(\lambda_{j,t})}|x_{j,t})
\label{eq:}
\end{equation}

where the summation is over all feasible values of $\Lambda_t$. The number of terms in this summation is $M_t (M_t-1) ... (M_t-K_t)$, which may be prohibitively large if there are many targets or observations in the scene.

If it is necessary to calculate the likelihood very often, as in a particle filter, it may be preferable to estimate $\Lambda_t$ instead of marginalising it.