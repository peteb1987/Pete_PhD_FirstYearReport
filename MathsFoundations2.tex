\subsection{The basic filter}

In a few simple cases the filtering set-up permits the derivation of closed form posterior distributions at each time instant. Most notably, the Kalman filter (KF) \cite{Kalman1960} is an analytic filter for models with continuous state and observation variables, in which both transition and observation densities are Gaussian with mean values expressed as linear transformations

\begin{equation}
P(x_t|x_{t-1}) = \mathcal{N}(x_t|Ax_{t-1}, Q)
\label{eq:LinearFilterEq1}
\end{equation}
\begin{equation}
P(y_t|x_t) = \mathcal{N}(y_t|C x_t, R)
\label{eq:LinearFilterEq2}
\end{equation}

where $Q$ and $R$ are covariance matrices. We use lower case variables here to emphasise that these are `nice', continuous vectors. (Later our state variables will be sets or lists).

Kalman's solution for the linear-Gaussian case is given by

\begin{equation}
P(x_t|y_{1:t-1}) = \mathcal{N}(x_t|\hat{\mu}_t, \hat{\Sigma}_t )
\label{eq:KFp}
\end{equation}
\begin{equation}
P(x_t|y_{1:t}) = \mathcal{N}(x_t|\mu_t, \Sigma_t )
\label{eq:KF}
\end{equation}

where $\mu_t$, $\Sigma_t$, etc. are given by the following recursions.

Time Updates:
\begin{equation}
\hat{\mu}_t = A \mu_{t-1}
\label{eq:KFTime1}
\end{equation}
\begin{equation}
\hat{\Sigma}_t = A \Sigma_{t-1} A^{T} + Q
\label{eq:KFTime2}
\end{equation}

Measurement Updates:
\begin{equation}
z_t = y_t - C \hat{\mu_t}
\label{eq:KFMeas1}
\end{equation}
\begin{equation}
S_t = C \hat{\Sigma}_t C^{T} + R
\label{eq:KFMeas2}
\end{equation}
\begin{equation}
K_t = \hat{\Sigma}_t C^{T} S_t^{-1}
\label{eq:KFMeas3}
\end{equation}
\begin{equation}
\mu_t = \hat{\mu}_t + K_t z_t
\label{eq:KFMeas4}
\end{equation}
\begin{equation}
\Sigma_t = (I - K_t C) \hat{\Sigma}_t
\label{eq:KFMeas5}
\end{equation}

The KF is delightful because it not only provides us with a closed-form analytic solution, but the complexity of that solution does not increase as we receive additional measurements. This is a consequence of the fact that the Gaussian distribution is its own conjugate prior. Unfortunately, no other such convenient cases have been discovered \cite{Daum2005}. Analytic solutions to nonlinear, non-Gaussian filtering problems generally require additional constraints, such as zero process noise $Q=0$ \cite{Daum2005}.

\subsection{The extended filter}

Given the efficiency of the KF, our instinct when faced by an intractable non-linear filtering problem is to linearise it. This produces the Extended Kalman Filter (EKF), and is achieved by replacing the $A$ and $C$ matrices in equations~\ref{eq:KFTime1} through~\ref{eq:KFMeas5} above with Jacobians

\begin{equation}
A_t = \left . \frac{\partial f}{\partial x_t} \right \vert _{\mu_{t-1}}
\label{eq:EKF1}
\end{equation}
\begin{equation}
C_t = \left . \frac{\partial g}{\partial x_t} \right \vert _{\hat{\mu}_t}
\label{eq:EKF2}
\end{equation}

where $f(x_{t-1})$ and $g(x_t)$ are the mean functions of the transition and observation models. For non-Gaussian distributions, the covariance matrices should also be matched.



\subsection{The Kalman smoother}

The KF gives us an optimum estimate of $P(x_t|y_{1:t})$. However, once more data has arrived, we can improve this estimate. For a given set of data, $y_{1:T}$, we can estimate the optimum estimates for all previous state distributions, $P(x_{t}| y_{1:T})$ using a Rauch-Tung-Striebel (RTS) smoother, \cite{Rauch1965}. This begins with a normal KF, followed by a backward filtering pass which propagates information to earlier time instances. This backward pass is implemented by the following recursions

\begin{equation}
\tilde{\mu}_t = \mu_{t} + \Sigma_t A^T \hat{\Sigma}_{t+1}^{-1} (\tilde{\mu}_{t+1} - \hat{\mu}_{t+1})
\label{eq:RTS_mean}
\end{equation}
\begin{equation}
\tilde{\Sigma}_t = \Sigma_{t} + [\Sigma_t A^T \hat{\Sigma}_{t+1}^{-1}] (\tilde{\Sigma}_{t+1} - \hat{\Sigma}_{t+1}) [\Sigma_t A^T \hat{\Sigma}_{t+1}^{-1}]^T
\label{eq:RTS_covar}
\end{equation}

giving us

\begin{equation}
P(x_t|Y_{1:T}) = \mathcal{N}(x_t|\tilde{\mu}_t, \tilde{\Sigma}_{t}).
\end{equation}

For a full derivation, see \cite{Rauch1965}. There exist other ways to implement Kalman smoothing in a fixed-interval sense, such as the forward-backward smoother, and in a fixed-lag sense, but they will not be used in this work.